\documentclass[a4paper]{article}

\usepackage{a4wide}
\usepackage{amssymb}
\usepackage{xcolor}
\usepackage{comment}
\usepackage{graphicx}
\usepackage[dutch]{babel}

\title{Elektronisch stemmen \\ \large De ethische discussie over elektronisch stemmen in de digitale samenleving.}
\author{
Mick van Gelderen \\ 4091566 \and 
Mick de Lange \\ 1534068 \and
Salim Salmi \\ 4089715
}


\newcommand{\TODO}[1]{{\color{red}\textbf{TODO: #1}}}

\usepackage{amsmath}
\begin{document}

\thispagestyle{plain}
\maketitle

\hfill \\ \\ \\ \\ \\ \\ \\ \\ \\ \\
\begin{figure}[htp]
\centering
\includegraphics[width=\textwidth]{media/voting_machines.png}
\label{fig:voting-machines}
\begin{comment}
Randall Munroe, (2013), Voting Machines [ONLINE]. Available at: http://imgs.xkcd.com/comics/voting_machines.png [Accessed 15 March 13].
\end{comment}
\end{figure}

\newpage

\thispagestyle{plain}

\section*{Voorwoord}
Dit artikel is geschreven door 3 studenten Technische Informatica aan de Technische Universiteit Delft voor het vak `informatietechnologie en waarden'.
Het belicht een ethische kwestie van verschillende standpunten aangesterkt met literatuur uit het vakgebied en de filosofie. 

\section*{Samenvatting}

\newpage

\thispagestyle{plain}
\renewcommand{\contentsname}{Inhoud} 
\tableofcontents

\newpage

\section{Inleiding}

In de afgelopen jaren is er in Nederland veel te doen geweest rondom elektronisch stemmen.
Zo zijn er een aantal proeven geweest, met verschillende vormen van elektronisch stemmen.
Er zijn naar aanleiding van een aantal problemen actiegroepen opgericht die zich duidelijk uitspraken tegen elektronisch stemmen.
De overheid heeft uiteindelijk verschillende commissies ingesteld die onderzoek hebben gedaan naar de problemen en naar de benodigdheden om in de toekomst elektronisch stemmen in te voeren.

Er spelen binnen dit onderwerp verschillende technische en morele onderwerpen.
Veel genoemd zijn privacy en veiligheid, maar ook de mogelijkheid om bijvoorbeeld gemakkelijker en sneller de stemmen te tellen.
Er zijn op het gebied van elektronisch stemmen dan ook veel problemen, maar ook voordelen.
Op het moment gebruiken we in Nederland weer het papieren stembiljet, maar er zijn ook landen waar wel elektronisch gestemd kan worden.
Wij willen in dit artikel in gaan op de verschillende aspecten en onderzoeken wat deze problemen en voordelen met zich mee brengen.

Allereerst zullen we hier onze onderzoeksvraag definiëren, dit wordt het uitgangspunt van dit paper en de leidraad van onze uiteindelijke conclusie.
Vervolgens zullen we de gebruikte definitie van elektronisch stemmen bepalen en de verschillende technische en morele aspecten die er spelen belichten.
Aan de hand van deze aspecten zullen we de argumenten voor elektronisch stemmen uiteenzetten.
Daarna zetten we de argumenten tegen elektronisch stemmen op een rijtje.
Uiteindelijk zullen we aan de hand van deze argumenten onze eigen positie in deze ethische discussie bepalen en toelichten hoe en waarom wij tot deze conclusie zijn gekomen.

\subsection{Onderzoeksvraag}

Onze onderzoeksvraag luidt: ``Zouden we over moeten stappen op elektronisch stemmen?''.

Om deze vraag te beantwoorden zullen de verschillende aspecten rondom elektronisch stemmen ter discussie moeten worden gesteld.
Het belichten van deze aspecten en beargumenteren van de pro en con argumenten zal dan uiteindelijk leiden tot een antwoord op deze vraagstelling.

Hierna zullen we eerst verder ingaan op de door ons gebruikte definitie van elektronisch stemmen en de aspecten die daarbij relevant zijn.

\section{Elektronisch stemmen}

Ten eerste zal wat meer inzicht worden gegeven over het elektronische stemmen. 
In dit onderdeel zullen eerst de actoren die een rol spelen worden voorgesteld. 
Daarna worden een aantal technische feiten beschreven zodat het duidelijk is waar elektronisch stemmen over gaat.
Tevens zullen de verschillende morele aspecten omringd het elektronisch stemmen worden beschreven.
Ten slotte zullen de juridische aspecten die bij het elektronische stemmen komen kijken worden besproken.

\subsection{Betrokken partijen}

De twee meest voor de hand liggende partijen bij verkiezingen zijn natuurlijk de politieke partijen en de burgerlijke kiezers. 
Beide hebben echter zeer verschillende belangen en standpunten. 

Fabrikanten van de stemapparatuur en software.
 

\subsection{Technische feiten}

Als men het over elektronisch stemmen heeft dan wordt bedoelt het op een over ander manier vastleggen van jouw stem vast te leggen op een digitaal systeem.
Hier is echter nog niet alles mee gezegd, want dit kan op een groot aantal verschillende manieren gebeuren en elk van deze technieken heeft een aantal gevolgen voor de actoren. 

De huidige manier van stemmen betreft het invullen van de keuze op een voorgedrukt papieren stembiljet. 
Echter kan er zelfs op dit niveau al techniek aan te pas komen door middel van het scannen van de keuze gemaakt op het stembiljet en deze elektronisch vast te leggen.
Een andere mogelijkheid van elektronisch stemmen is het gebruik van een elektronisch stemapparaat met daarnaast een papieren stembiljet voor controle.
Dit zijn allemaal manieren om elektronisch stemmen te combineren met het papieren stem systeem.
Het voornaamste doel hier is het vergemakkelijken van de stemmentelling.


Stemmen op een stemcomputer werd tot op kort in Nederland gebruikt.

Stemmen op het internet

\subsection{Morele aspecten}
Om te bepalen welke morele aspecten er spelen rondom elektronisch stemmen, zullen we eerst moeten kijken naar de geldende normen en waarden rondom het uitbrengen van een stem.
In de huidige vorm van stemmen, met papieren biljetten, zijn een aantal waarden die een belangrijke rol spelen.
Niet al deze waarden hebben altijd een dergelijke rol gehad bij eerdere stemmethoden.
Bij het invoeren komen van elektronisch stemmen zijn er ook nog andere waarden die een rol gaan spelen, deze komen hier ook aan bod.

\subsubsection{Privacy}
In het huidige kiessysteem is het erg belangrijk dat we onze daadwerkelijke stem voor ons zelf houden.
Ondanks dat iedereen kan vertellen wat hij heeft gestemd, hoeft dit niet de waarheid te zijn en kan iedereen de stem uitbrengen die hij zelf wil.
Deze norm is ontstaan uit de angst voor afpersing of bedreiging, waardoor een kiezer gedwongen kan worden anders te stemmen dan zijn eigen voorkeur.
Omdat dit als een zodanig belangrijk punt wordt gezien is deze norm ook opgenomen in de wetgeving, hierover meer in het onderdeel `Juridische aspecten'.

Overigens speelde privacy niet altijd een dergelijke rol bij het uitbrengen van stemmen.
Zo werd er vroeger ook wel mondeling gestemd, waarbij een persoon fysiek in een publieke ruimte zijn stem uitsprak.
Hierbij was het dus mogelijk om precies te zien wie wat stemde, iets wat in die periode juist als een belangrijke waarde werd gezien.

\subsubsection{Vrijheid}
Vrijheid is enigszins gekoppeld aan de voorgaande waarde, het gaat hier namelijk om de vrijheid om zelf te bepalen welke partij jouw stem verdient.
Deze waarde staat ook weer in verband met het probleem van afpersing of bedreiging.
Vrijheid zou gezien kunnen worden als de onderliggende waarde voor de norm over privacy in het uitbrengen van een stem.
Om een democratisch systeem te laten werken vinden mensen het van belang dat ieder persoon kan stemmen op de persoon of partij van zijn eigen voorkeur, zonder dat invloeden van buiten af effect hebben op deze keuze.

Ook zijn er natuurlijk andere factoren die invloed kunnen uitoefenen op de stemvoorkeur van een persoon, zoals media en campagne voering, of bijvoorbeeld groepsdruk.
Deze factoren spelen weliswaar een rol in de keuzevrijheid van het individu, maar zijn even belangrijk bij zowel het huidige papieren stemmen als elektronisch stemmen.
Hier zullen we dan ook niet al te ver op in gaan in dit paper.

\subsubsection{Eerlijkheid en vertrouwen}
Bij verkiezingen zijn eerlijkheid en vertrouwen ook belangrijke waarden , die dicht bij elkaar liggen.
Hierbij gaat het dan met name over het vertrouwen van de kiezer in de eerlijkheid van het tellen van de stemmen.
Stemmen tellen gebeurt in de huidige papieren vorm, met de hand.
Deze mensen die de telling uitvoeren controleren elkaar, om grote fouten te voorkomen.
In verkiezingen vinden we het belangrijk dat elke partij of kiespersoon de stemmen krijgt die ook daadwerkelijk zijn uitgebracht op die persoon of partij.

Vertrouwen is heel duidelijk een waarde die in alle vormen van stemmen terug komt.
Zowel in oudere varianten als in elektronisch stemmen speelt dit een rol.
Immers om een juiste verkiezingsuitslag te garanderen is een eerlijke, betrouwbare telling van groot belang.

Door middel van openheid in het proces van stemmen tellen probeert men de eerlijkheid te garanderen.
Het proces moet een open karakter hebben, zodat elke burger begrijpt en ziet hoe de stemmen worden geteld.
Dit begrip en inzicht is voor de burger die zijn stem uitbrengt essentieel om vertrouwen te hebben in het proces.

\subsection{Juridische aspecten}
Rondom het uitbrengen van stemmen zijn ook een aantal belangrijke juridische aspecten die een rol spelen.
Deze liggen vast gelegd in het wetboek, specifiek het artikel van de kieswet.
De kieswet omvat een aantal wetgevingen die belangrijke normen en richtlijnen over het uitbrengen van stemmen vastleggen.
Deze wetgevingen richten zich er op het stemproces zo goed mogelijk te laten verlopen en een juiste uitslag te garanderen.

Ook worden er in de kieswet veel randzaken vastgelegd, om staatkundig de rechtsgeldigheid van een stem te garanderen.
Dit laatste onderwerp zullen wij hier minder belichten, gezien dit een nauwkeurige omschrijving van het stembiljet betreft, waar bij invoering van elektronisch stemmen de wetgeving uiteraard aanpassingen behoeft.
Overigens dient opgemerkt te worden dat er bij de kortstondige invoering van elektronisch stemmen een artikel aan de wetgeving was toegevoegd die dit mogelijk maakte, welke op dit moment uit de wetgeving is geschrapt.

Wij zullen hier alleen een aantal juridische aspecten toelichten die relevant zijn voor de vraagstelling die in dit paper wordt behandeld.

\subsubsection{Kiesgerechtigden}
Kiesgerechtigden zijn de personen die gerechtigd zijn om bij verkiezingen hun stem uit te brengen.
De wet schrijft voor dat deze personen dienen te worden geïdentificeerd als zodanig bij het uitbrengen van hun stem.
Deze wet voorkomt dat personen die niet gerechtigd zijn om een stem uit te brengen, dit alsnog kunnen doen.
Op deze manier wordt gecontroleerd dat alleen de juiste personen voor de juiste verkiezingen hun stem kunnen uitbrengen.

Hierdoor wordt ook controle gehouden op het aantal keren dat een individu zijn stem kan uitbrengen.
Door de identificatie en het inleveren van de stempas bij het stembureau wordt gecontroleerd dat deze persoon slechts eenmaal kan stemmen.
Wanneer het mogelijk zou worden meerdere malen te stemmen kunnen kwaadwillenden uiteraard de uitslag beïnvloeden.

Voor onze vraagstelling is zowel het onderdeel dat personen altijd slechts eenmaal hun stem kunnen uitbrengen relevant, als het feit dat de juiste persoon de stem uitbrengt. 
Deze twee onderdelen van dit juridische aspect spelen een belangrijke rol in de discussie rond onze vraagstelling over de invoer van elektronisch stemmen.

Wie er precies wel of niet het stemrecht zouden moeten hebben is weer een aparte discussie, hier zullen we in dit paper dan ook niet verder op in gaan.

\subsubsection{Stemgeheim}
Zoals eerder besproken vinden we het een belangrijke norm dat iemand zijn stemvoorkeur voor zichzelf kan houden.
Deze waarde wordt dusdanig belangrijk gevonden dat het bij wet is vastgelegd.
Deze wet beschermd kiezers voor mogelijke druk van buitenaf om een andere stem uit te brengen dan de persoon zelf zou willen.
De wet verbiedt zelfs de mogelijkheden om bewijs te geven van de stemkeuze.
Dit houdt in dat een persoon wel mag vertellen wat hij gestemd heeft, maar deze persoon behoudt altijd de mogelijkheid hier oneerlijk over te zijn en heeft geen mogelijkheid om te bewijzen of datgene ook juist is.

Deze wetgeving betekent in de praktijk dus dat het niet is toegestaan om samen met iemand anders een stem uit te brengen.
Vandaar ook de verplichting van maximaal één persoon in een stemhokje.
Dit is een aspect dat interessant is voor onze vraagstelling, omdat bij sommige vormen van elektronisch stemmen het moeilijk is om deze wet te handhaven.

\section{Waarom wel elektronisch stemmen?}

\subsection{Modernisering}

\subsection{Nauwkeurigheid}

\subsection{Gemak en snelheid}

\section{Waarom niet elektronisch stemmen?}

\subsection{Betrouwbaarheid}

\subsection{Veiligheid}

\subsection{Privacy}

\section{Conclusie}

Hier komen we tot een conclusie.

\newpage

\TODO{wetboek van strafrecht reference toevoegen: kieswet.}

\bibliographystyle{plain}
\renewcommand\refname{Literatuur}
\bibliography{references}

\end{document}








